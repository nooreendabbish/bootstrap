% Created 2015-04-13 Mon 16:24
\documentclass[11pt]{article}
\usepackage[utf8]{inputenc}
\usepackage[T1]{fontenc}
\usepackage{fixltx2e}
\usepackage{graphicx}
\usepackage{longtable}
\usepackage{float}
\usepackage{wrapfig}
\usepackage{soul}
\usepackage{textcomp}
\usepackage{marvosym}
\usepackage{wasysym}
\usepackage{latexsym}
\usepackage{amssymb}
\usepackage{hyperref}
\tolerance=1000
\providecommand{\alert}[1]{\textbf{#1}}

\title{Recent Developments}
\author{Nooreen Dabbish, Daqian Huang, and Shinjini Nandi}
\date{\today}
\hypersetup{
  pdfkeywords={},
  pdfsubject={},
  pdfcreator={Emacs Org-mode version 7.9.3f}}

\begin{document}

\maketitle


\begin{abstract}

\end{abstract}

\section{Instructions}
\label{sec-1}

Evaluation: You will be evaluated on the basis of BOTH the quality and
clarity of in-class presentation and the final report. All the
students of any group will get the SAME grade (depending on their
contribution on the project). Before presentation you MUST TURN IN the
printed final report to me.
 
Format of the Report: Clear Title, Names of Teammates (and their
individual contributions) and Abstract on the first page. Summary of
the main idea in the remaining 3-4 pages. The LAST section of your
report should discuss what are the possible ways you think this work
could be further extended. Use Latex to prepare the manuscript. Clear
reference list should be provided. While writing the manuscript your
emphasize will be to summarize the main findings in your own words (do
not copy paste). Presentation should be simple yet comprehensive. You
are highly recommended to use numerical examples (when applicable) to
better illustrate the ideas. 
 
Format of presentation: Board work. Use your manuscript while
presenting. Time: 40 mins + 5-7 mins Q\&A session. . Illustration using
data set and case studies (while presenting your idea) is highly
valuable. 
 
Start as soon as possible with your group members.
 
This is a GREAT opportunity. Successful projects in the past have gone
on to become full-fledged research papers !
\section{Recent Developments in Bootstrap Methodology}
\label{sec-2}
\subsection{Introduction}
\label{sec-2-1}

 
This article sets out to ``give a bird's eye overview of the current
state of bootstrap research.'' The authors cover basic ideas with
references to bootstrap literature as well as giving in-depth
explanation and examples of extensions. Topics include parametric
inference using bootstrap stimulations, non-uniform nonparametric
sampling, bootstrap failure, hypothesis testing, bagging, dependent
data and other topics. Our goal is to summarize and highlight the
examples and ideas in the paper, and we conclude with a section on
future directions suggesting what next steps can take this work further.
\subsection{Basic Ideas: Bootstrap approaches to confiendence intervals and hypothesis testing}
\label{sec-2-2}

 
\begin{itemize}
\item Nonparametric confidence intervals
\begin{itemize}
\item either Studentized pivots or direct use of quantiles
\item Studentized bootstrap uses estimated var V$^{\star}$ are second-order accurace 1-$\alpha$+O(n$^{\mathrm{-1}}$)
\item improvement over O(n$^{\mathrm{-1/2}}$)
\item BC$_a$ also second-order but additionally tranformation-invariant.
\end{itemize}
\item Using a pivot
\begin{itemize}
\item avoids the need to modify the samping plan in hypothesis testing
\end{itemize}
\item Model-based bootstrapping
\begin{itemize}
\item when data are not identically distributed.
\item time-series/autoregressive moving average
\item Notes on Notation:
\begin{itemize}
\item $\hat{F}$ empirical distribution
\item F(y;$\psi$) parametric model with parameter $\psi$
\item 
\end{itemize}
\item Topics to explore/look-up
\begin{itemize}
\item conditions under which bootstrap is consistent (Bickel and
    Freedman 1981)
\item Edgeworth correction
\item Edgeworth expansion
\item permutation tests
\end{itemize}
\end{itemize}
\end{itemize}
\subsection{Bootstraps for Parametric Likelihood Inference}
\label{sec-2-3}


\begin{itemize}
\item profile log-likelihood l$_p$($\gamma$)
\item ratio statistic $w_p(\gamma) = 2(l_p(\hat{\gamma})-l_p(\gamma)) \sim
  \chi^2_1 + O(n^{-1})$
\item signed root likelihood ratio statistic $$r_p =
  sgn(\hat{\gamma}-\gamma)w_p(\gamma)^{1/2} \sim N(0,1) +
  O(n^{-1/2})$$
\item r$_a$ adjusted r$_p$
\end{itemize}
\subsubsection{Example 1 Exponential Regression}
\label{sec-2-3-1}

Lawless(1982)


\begin{center}
\begin{tabular}{rr}
 Survival in weeks  &  log wbc  \\
\hline
               156  &     2.88  \\
               108  &     4.02  \\
               143  &     3.85  \\
                56  &     3.97  \\
                 1  &      5.0  \\
\hline
\end{tabular}
\end{center}
\subsection{Weighted Non-parametric Bootstrapping}
\label{sec-2-4}
\subsection{Subsampling and the m out of n bootstrap}
\label{sec-2-5}
\subsection{Bootstrapping Superefficient estimators}
\label{sec-2-6}
\subsection{More on Significance Tests}
\label{sec-2-7}
\subsection{Bagging and Classification}
\label{sec-2-8}
\subsection{Bootstrapping Dependent Data}
\label{sec-2-9}
\subsection{Other topics}
\label{sec-2-10}
\subsection{Final Remarks}
\label{sec-2-11}
\section{Future Directions}
\label{sec-3}

\end{document}
